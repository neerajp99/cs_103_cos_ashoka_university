\documentclass[11pt]{article}
\usepackage{amsmath,textcomp,amssymb,geometry,graphicx,enumerate}
\usepackage{algorithm} % Boxes/formatting around algorithms
\usepackage[noend]{algpseudocode} % Algorithms
\usepackage{hyperref}
\hypersetup{
    colorlinks=true,
    linkcolor=blue,
    filecolor=magenta,      
    urlcolor=blue,
}

\def\Name{Neeraj Pandey}  % Your name
%\def\SID{PUT SOMETHING HERE}  % Your student ID number
%\def\Login{PUT SOMETHING HERE} % Your login (your class account, cs170-xy)
\def\Homework{2} % Number of Homework
\def\Session{Monsoon 2018}


\title{CS103 -- Monsoon 2018 --- Homework \Homework}
\author{\Name 
%SID \SID, 
%\texttt{\Login}
}
\markboth{CS103\Session\  Homework \Homework\ \Name}{CS103--\Session\ Homework \Homework\ \Name, %\texttt{\Login}
}
\pagestyle{myheadings}
\date{}

\newenvironment{qparts}{\begin{enumerate}[{(}a{)}]}{\end{enumerate}}
\def\endproofmark{$\Box$}
\newenvironment{proof}{\par{\bf Proof}:}{\endproofmark\smallskip}

\textheight=9in
\textwidth=6.5in
\topmargin=-.75in
\oddsidemargin=0.25in
\evensidemargin=0.25in


\begin{document}
\maketitle

Collaborators: NONE

\section*{1. Express the following hexadecimal number in decimal and binary forms: 0xd3fa986.}
\begin{qparts}

\textbf{Solution: } Given hexadecimal code is: d3fa986 (neglecting the hex notation 0x)
We know the decimal value of the hexadeimal characters; d,f,a:
\[d=13, f=15, a=10\]
The decimal value for the hexadecimal \textbf{d3fa986}: \[\small( 6*16^0\small) + \small( 68*16^1\small) + \small( 9*16^2\small) + \small( 10*16^3\small) + \small( 15*16^4\small) + \small( 3*16^5\small) + \small( 13*16^6\small)\]
\[\implies 222275974\]
\newline
Now, we have to convert decimal into binary
(`r is the remainder here`): 
\[= 222275974_{10} / 2 - r_0\]
\[= 111137987_{10} / 2 - r_1\]
\[= 55568993_{10} / 2 - r_1\]
\[= 27784496_{10} / 2 - r_0\]
\[...\]
\[...\]
\newline
Continuing the process until we have reached 0.
\[\implies 1101001111111010100110000110_{2}\]

\end{qparts}


\section*{2. Annotate the following MIPS instructions to indicate source and destination registers. Also, describe in words (one sentence or less) what the instruction is trying to do. }

\begin{qparts}
\item addi \$t1, \$zero, 8

\textbf{Solution:} Destination: \$t1 \newline
Source: \$zero
\newline 
Instruction: To get the value of \$t1 by performing the calculation of 0+8

\item lw \$s3, 4(\$gp)
\textbf{Solution:}Destination: \$s3
\newline
Source: \$gp
\newline
Instruction: To get the value of \s3 from the base address of 4.


\item sw \$s4, 10(\$s5)
\textbf{Solution:}Destination: \$s5
\newline
Source: \$s4
\newline
Instruction: Store the value of \$s4 in a specified address.

\end{qparts}
\end{document}
